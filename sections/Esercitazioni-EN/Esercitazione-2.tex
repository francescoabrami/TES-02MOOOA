\documentclass[../main.tex]{subfiles}

\graphicspath{{\subfix{../images/}}}

\newpage


\begin{document}

%% BEGIN WRITING %%

\subsection{Esercitazione 2 - 19/11/2025}

In questa seconda esercitazione si % TODO FINIRE DESCIRIZONE

\subsubsection{Esercizio 1 - DTFT di un segnale}

Si calcoli la DTFT del segnale:

\[ x[n] = u[n] - u[n-10] + n \cdot e^{-n}u[n] \]\

Cominciamo con la risoluzione dell'esercizio andando, come prima cosa, a disegnare il segnale $x[n]$ formato da tre parti che tratteremo in modo separato in quanto vale la proprietà di linearità. 
In particolare i tre pezzi che compongono $x[n]$ sono un gradino $u[n]$, un gradino ritardato di dieci campioni ed infine una sequenza esponenziale decrescente non nulla solo per valori positivi e moltiplicata a sua volta per un valore $n$.
La differenza delle prime due sequenze genera, come abbiamo già visto, una porta definita tra l'origine ed $n = 9$.\\

Possiamo ora, per linearità, andare a trasformare i singoli pezzi per poi ricomporli più tardi.
In particolare denominando con $x_1[n]$ la porta definita dalla prime due sequenze e $x_2[n]$ la restante parte otteniamo che:

\[ X_1(e^{j2 \pi f}) = \sum_{n = -\infty}^{+\infty} x_1[n] \cdot e ^{-j2 \pi f n}  \]

Posso riscrivere l'espressione appena vista come:

\[ \sum_{n = 0}^9 e^{-j2\pi n f} = \frac{1 - e^{-j2\pi f \cdot 10}}{1 - e^{-j2\pi f}}\]\

Arrivati a questo punto possiamo trasformare il secondo pezzo denominato $x_2[n]$.
In particolare scrivendo $x_2[n]$ come:

\[ x_2[n] = n \cdot z[n] \hspace{15 pt} \unit{con} \hspace{15 pt} z[n] = e^{-n} \cdot u[n]  \]

Possiamo ottenere il risultato appena visto in quanto a lezione % TODO PARTE CHE MANCA

Passando alla seconda parte, ovvero $x_2[n]$, possiamo andare ad applicare la proprietà della derivazione in frequenza secondo la seguente formula dove:

\[ n \cdot x[n] \rightarrow \frac{j}{2\pi} \cdot \diff{}{f} X(e^{j2\pi f}) \]\

Possiamo quindi calcolare la trasformata del termine $e^{-n} \cdot u[n]$ parte ottenendo:

\[ \sum_{n = -\infty}^{+\infty}  e^{-n} \cdot u[n] \cdot e^{-j2\pi fn} \]\

Sapendo che moltiplico il tutto per un gradino unitario limito la mia sommatoria ad indici positivi:

\[ \sum_{n = 0}^{+\infty}  e^{-n} \cdot e^{-j2\pi fn} \]\

utilizzando le proprietà degli esponenti ottengo che:


\[ \sum_{n = 0}^{+\infty}  e^{-n} \cdot e^{-j2\pi fn}  = \sum_{n = 0}^{+\infty}  e ^{-n \cdot (2j\pi f + 1)}\]\

In definitiva arriviamo al risultato seguente attraverso % TODO CAPIRE COSA FATTO, SERIE?


\[ \sum_{n = 0}^{+\infty}  e ^{-n \cdot (2j\pi   + 1)} = \frac{1}{1 - e ^{-(2j\pi f + 1)} }\]\

Ora scomponendo l'esponenziale al denominatore ed applicando la proprietà sopracitata si ottiene che:

\[ \frac{j}{2\pi} \cdot \diff{}{f} X(e^{j2\pi f}) = \frac{j}{2\pi} \cdot \diff{}{f} \frac{1}{1 - e^{-1} e^{-2j\pi f}}\]\

Ora non ci basta che calcolare la derivata come indicato di seguito, utilizzando le opportune tecniche di derivazione, in questo caso si utilizza la regola di derivazione di un rapporto.

\[ \diff{}{x} \frac{f(x)}{g(x)} \rightarrow \frac{f'(x) \cdot g(x) - f(x) \cdot g'(x)}{\big( g(x) \big)^2} \]\

Segue lo svolgimento del calcolo:

\[ \frac{j}{2\pi} \diff{}{f} \frac{1}{1 - e^{-1} e^{-2j\pi f}} = \frac{j}{2\pi} \frac{e^{-1} \cdot e^{-j2\pi f} \cdot (-j2\pi)}{(1 - e^{-1} e^{-2j\pi f})^2} \]\

A questo punto il denominatore resta invariato, se non per la semplificazione di $2\pi$ con il numeratore, mentre al numeratore avviene la semplificazione di $-j \cdot j$ ad 1.
Otteniamo in definitiva il seguente risultato:

\[ \frac{e^{-1} \cdot e^{-j2\pi f}}{(1 - e^{-1} e^{-2j\pi f})^2} \]\

Arrivati a questo risultato il risultato atteso, ovvero la trasformata di $x[n]$, è stato raggiunto concludendo di fatto l'esercizio.

\subsubsection{Esercizio 2 - Costruzione di una sequenza e calcolo DTFT}

Si consideri una sequenza $x[n]$ con DTFT pari a $X(e^{j2 \pi f})$. Si costruisca una sequenza $y[n]$ a partire da $x[n]$ con la regola:

\[ y[2n] = x[n] \]
\[ y[2n+1] = -x[n] \]\

Si calcoli ora la DTFT di $y[n]$.\\

Iniziamo la risoluzione dell'esercizio notando come la regola descritta sopra opera su una sequenza qualsiasi $x[n]$ ritornando il valore assunto della sequenza nel caso in cui il campione sia pari mentre il valore cambiato di segno nel caso in cui il campione sia dispari.
Compreso ora come opera la nostra regola possiamo andare avanti considerando la generica sequenza $x[n]$ come la sommatoria di due sequenze, una a termini pari ed una a termini dispari, separate tra di loro.\\



\subsubsection{Esercizio 3 - Analisi FFT su un segnale limitato nel tempo}

Un segnale praticamente limitato nel tempo per $0 \leq t \leq T_1$ con $T_1$ = 1s e limitato in banda per $|f| \leq B_x$ con $B_x$ = 32 Hz viene campionato alla
frequenza di Nyquist $\tfrac{1}{T_0} = 2B_x$. \\ I campioni $x_n = x(nT_0)$, dove $T_0$ è il periodo di campionamento, vengono usati per valutare numericamente lo spettro del segnale mediante una FFT a radice due. Si richiede una risoluzione in
frequenza $\Delta f$ = 0.5 Hz. Dire quale delle seguente affermazioni è corretta:

\begin{enumerate}
	\item[\textbf{1)}] Per conseguire la risoluzione in frequenza richiesta è necessario elaborare almeno $N$ = 128 campioni e può essere necessario estendere il segnale nel tempo con campioni nulli.
	\item[\textbf{2)}] Non è possibile in alcun modo conseguire la risoluzione in frequenza richiesta.
	\item[\textbf{3)}] Non è possibile conseguire la risoluzione in frequenza richiesta se non campionando il segnale ad una frequenza superiore alla frequenza di Nyqvist.
	\item[\textbf{4)}] Nessuna delle altre risposte è corretta.
\end{enumerate}







\subsubsection{Esercizio 4 - Sequenza campionata e DFT}

Si consideri la sequenza $x[n]$ di $N = 10$ campioni che vale 1 per $n = 0,2,8$
e zero altrove. Si calcoli $X[k] = \unit{DFT}\{x[n]\}$.

\subsubsection{Esercizio 5 - Spettro e campionamento di un segnale attraverso FFT}

Si consideri un segnale $x(t)$ il cui spettro $X(f)$ è nullo per $|f| > f_x$, con $f_x = 10$ Hz. Si costruisca il segnale $y(t) = x(t) \cdot \cos(2\pi f_y t)$ con $f_y = 50$ Hz. Si vuole valutare lo spettro $Y(f)$ a partire da opportuni campioni di $y(t)$, usando una FFT a radice 2. Volendo ottenere una risoluzione in frequenza di 3 Hz, si valuti il passo di campionamento da scegliere per $y(t)$, il numero di campioni $N$ e la risoluzione finale ottenuta.

\subsubsection{Esercizio 6 - Costruzione di una sequenza e DFT}

A partire dal segnale analogico $x(t) = A_1 \cdot \cos(2\pi f_0 t) + A_2 \sin(2\pi f_0 t)$ (con $A_1$ e $A_2$ costanti positive) si costruisce la sequenza $x[n] = x(nT_c)$ con $T_c = 1/(2f_0)$. Si considerino $N$ = 10 campioni di $x[n]$ nell’intervallo $0 \leq n \leq 9$ e la sequenza $X[k] = \unit{DFT}\{x[n]\}$. Dire quale delle seguenti affermazioni è falsa:

\begin{enumerate}
	\item[\textbf{1)}] La sequenza campionata vale $x[n] = A_1e^{j\pi n}$
	\item[\textbf{2)}] $X[k]$ = 0 per $0 \leq k \leq 5$
	\item[\textbf{3)}] $X[k]$ = 0 per $0 \leq k < 5$
	\item[\textbf{4)}] $X[k]$ = $10 \cdot A_1$ per $k = 5$
\end{enumerate}








\end{document}