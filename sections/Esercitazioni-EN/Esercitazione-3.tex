\documentclass[../main.tex]{subfiles}

\graphicspath{{\subfix{../images/}}}

\newpage


\begin{document}

%% BEGIN WRITING %%

\subsection{Esercitazione 3 - 27/11/2025}

\subsubsection{Esercizio 1 - Trasformata zeta razionale}

Si consideri un segnale a tempo discreto $x[n]$ che abbia una trasformata zeta
$X(z)$ razionale. Dire quale delle seguenti affermazioni è vera:

\begin{enumerate}
	\item[\textbf{1)}] Per un segnale $x[n]$ causale la regione di convergenza è l’interno di una circonferenza il cui raggio è pari al polo di modulo minimo.
	\item[\textbf{2)}] Per un segnale $x[n]$ causale la regione di convergenza è l’interno di una circonferenza il cui raggio è pari al polo di modulo massimo.
	\item[\textbf{3)}] Per un segnale $x[n]$ anti-causale la regione di convergenza è l’interno di una circonferenza il cui raggio è pari al polo di modulo massimo.
	\item[\textbf{4)}] Per un segnale $x[n]$ anti-causale la regione di convergenza è l’interno di una circonferenza il cui raggio è pari al polo di modulo minimo.
\end{enumerate}


\subsubsection{Esercizio 2 - Trasformata zeta e regione di convergenza}

Calcolare la trasformata zeta e la regione di convergenza dei seguenti segnali discreti:

\begin{enumerate}

	\item[\textbf{1)}] $x[n] = \alpha^{|n|}$
	\item[\textbf{2)}] $x[n] = \begin{cases}
		1 \hspace{10pt} 0 \leq n  \leq N-1 \\
		0 \hspace{10pt} \unit{altrove}
	\end{cases} $
	\item[\textbf{3)}] $x[n] = \begin{cases}
		1 \hspace{10pt} 0 \leq n  \leq N \\
		2N - n \hspace{10pt} N + 1 \leq n  \leq 2N \\
		0 \hspace{10pt} \unit{altrove}
	\end{cases} $
	
\end{enumerate}

\subsubsection{Esercizio 3 - Calcolo della trasformata zeta I}

Calcolare la trasformata zeta delle sequenze calcolando gli zeri ed i poli per ognuna di esse:

\begin{enumerate}
	\item[\textbf{1)}] $x[n] = \alpha^n \cdot u[n]$
	\item[\textbf{2)}] $x[n] = \sin[\omega_0n] \cdot u[n]$
	\item[\textbf{3)}] $x[n] = \cos[\omega_0n] \cdot u[n]$
	\item[\textbf{4)}] $x[n] = \alpha^n \cdot \cos[\omega_0n] \cdot u[n]$
	\item[\textbf{5)}] $x[n] = n\alpha^n \cdot u[n]$
	\item[\textbf{6)}] $x[n] = n^2\alpha^n \cdot u[n]$
\end{enumerate}


\subsubsection{Esercizio 4 - Calcolo della trasformata zeta II}

Sia data la sequenza $x[n] = (- a)\cdot n \cdot u[n]$ con $u[n]$ la sequenza gradino unitario
e $ a = 0.5$. La trasformata z di $x[n]$, $X(z)$:

\begin{enumerate}
	\item[\textbf{a)}] Non ha poli
	\item[\textbf{b)}] Non ha zeri e ha due poli semplici reali in $z = \pm 0.5$
	\item[\textbf{c)}] Ha uno zero nell’origine, uno zero reale in $z = -0.5$ e due poli complessi coniugati in $z = \pm j0.5$
	\item[\textbf{d)}] Ha uno zero nell’origine e due poli complessi coniugati in $z = \pm j0.5$
	\item[\textbf{e)}] Ha uno zero nell’origine e un polo reale semplice in $z=-0.5$
\end{enumerate}


\subsubsection{Esercizio 5 - Calcolo della trasformata zeta e regione di convergenza}

Calcolare la trasformata zeta e la regione di convergenza dei seguenti segnali discreti:

\begin{enumerate}
	\item[\textbf{1)}] $x[n] = \big[ \big(\frac{1}{2}\big)^n + \big(\frac{3}{4}\big)^n\big] \cdot u[n-10]$
	\item[\textbf{2)}] $x[n] = \begin{cases}
		1 \hspace{10pt} -10 \leq n \leq 10 \\
		0 \hspace{10pt} \unit{altrove}
	\end{cases}$
\end{enumerate}

\subsubsection{Esercizio 6 - Calcolo sequenza casuale associata alla trasformata zeta}

Determinare le sequenze casuali associate alle seguenti trasformate zeta:

\begin{enumerate}
	\item[\textbf{1)}] $X_a(z) = \frac{1}{1 - \frac{1}{2}z^{-1}}$
	\item[\textbf{2)}] $X_b(z) = \frac{1}{(1 - \frac{1}{2}z^{-1})(1-z^{-1})}$
	\item[\textbf{3)}] $X_c(z) = \frac{1 - \frac{1}{2}z^{-1}}{1 + \frac{1}{2}z^{-1}}$
\end{enumerate}


\subsubsection{Esercizio 7 - Calcolo sequenza corrispondente alla trasformata zeta}

Calcolare la sequenza corrispondente alle seguenti trasformate zeta:

\begin{enumerate}
	\item[\textbf{1)}] $X(z) = (1 + 2z) \cdot (1 + 3z^{-1}) \cdot (1 - z^{-1}) $
	\item[\textbf{2)}] $X(z) = \frac{3z}{(z - \frac{1}{2})(z + \frac{1}{4})} \hspace{10pt} |z| > \frac{1}{2}$ 
	\item[\textbf{3)}] $X(z) = \frac{2z^3 + z^2}{(z+3)(z-1)}  \hspace{10pt} |z| > 3$
\end{enumerate}










\end{document}