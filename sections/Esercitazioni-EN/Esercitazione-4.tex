\documentclass[../main.tex]{subfiles}

\graphicspath{{\subfix{../images/}}}

\begin{document}

\newpage

%% BEGIN WRITING %%

\subsection{Esercitazione 4 - 4/12/2025}


\subsubsection{Esercizio 1 - Risposta all’impulso del filtro numerico}

Calcolare la risposta all’impulso del filtro numerico specificato dalla seguente
equazione ricorsiva:

\[ y[n] = x[n] - x[n-1] + \frac{3}{2} \cdot y[n - 1]\]\

\subsubsection{Esercizio 2 - Analisi filtro FIR I}

Dato il filtro FIR:

\[ y[n] = x[n] - x[n-4]  \]\

\begin{enumerate}
	\item[\textbf{1)}] Calcolare e disegnare il modulo e la fase della funzione di trasferimento $H(f) = H(e^{2j\pi f})$
	\item[\textbf{2)}] Calcolare la sequenza in uscita dal filtro quando in ingresso abbiamo la sequenza: \[ x[n] = \cos\bigg[ \frac{\pi}{2} n \bigg] + \cos\bigg[ \frac{\pi}{4} n \bigg] \]\
	\item[\textbf{3)}] Giustificare il risultato di \textbf{2)} utilizzando il risultato di \textbf{1)}
\end{enumerate}

\subsubsection{Esercizio 3 - Analisi sistema LTI a tempo discreto I}


Si consideri un sistema LTI a tempo discreto, descritto dalla seguente relazione ingresso/uscita:

\[ y(n) = \alpha x(n - 1) + 2\beta y(n - 1) - \beta^2 y (n - 2) \]\

dove $\alpha$ e $\beta$ sono numeri reali.


Si richiede di:

\begin{enumerate}
	\item[\textbf{1)}] Disegnare lo schema circuitale del sistema.
	\item[\textbf{2)}] Calcolare la funzione di trasferimento $H(z)$ e discutere la stabilità del sistema al variare dei parametri $\alpha$ e $\beta$.
	\item[\textbf{3)}] Calcolare la risposta all’impulso $h(n)$ e la risposta in frequenza $H(e^{j2\pi f})$.
	\item[\textbf{4)}] Ponendo $\alpha = 1$ e $\beta = \sqrt(2)$, calcolare l’uscita l’uscita $y(n)$ quando all’ingresso è posto il segnale $x(n)$ ottenuto dal campionamento della sinusoide analogica $x(t) = cos(2\pi f_0t)$ con frequenza di campionamento pari al quadruplo della frequenza di Nyquist.
\end{enumerate}


\subsubsection{Esercizio 4 - Analisi sistema LTI a tempo discreto II}

Si consideri il sistema LTI discreto con la seguente relazione tra ingresso e uscita:

\[  y[n] - \frac{1}{4} y[n - 1] = x[n] + x[n - 1] - 2x[n - 2] \]\

Si richiede di:

\begin{enumerate}
	\item[\textbf{1)}] Studiare poli e zeri della funzione di trasferimento.
	\item[\textbf{2)}] Dire se il sistema è di tipo FIR o IIR.
	\item[\textbf{3)}] Dire se il filtro è a fase minima.
\end{enumerate}

\subsubsection{Esercizio 5 - Analisi filtro FIR II}

Si consideri il filtro di tipo FIR con $H(z)$ come riportata di seguito.

\[H(z) = (1 - 0.9e^{j0.6\pi} z^{-1})(1 - 0.9 e^{-j0.6\pi} z^{-1} )(1 - 1.25 e^{j0.8\pi} z^{-1})(1 - 1.25 e^{-j0.8\pi} z^{-1}) \]\

Si richiede di:

\begin{enumerate}
	\item[\textbf{1)}] Studiare poli e zeri di $H(z)$.
	\item[\textbf{2)}] Studiare la stabilità del sistema inverso ${H^{-1}(z)}$.
\end{enumerate}

\subsubsection{Esercizio 6 - Analisi funzione di trasferimento}

Data la seguente funzione di trasferimento:

\[ H(z) = \frac{z^{-1} - \frac{1}{3}}{1 - \frac{1}{3} z^{-1}} \]\


Si richiede di:

\begin{enumerate}
	\item[\textbf{1)}] Studiare zeri e poli e discutere la stabilità.
	\item[\textbf{2)}] Calcolare il modulo della funzione di trasferimento $\big|H(e^{j2\pi f})\big|$
\end{enumerate}

\subsubsection{Esercizio 7 - Funzione di trasferimento dato un segnale}

La funzione di trasferimento di un sistema numerico vale:

\[ H(z) = \frac{1 - z^{-1} }{1 + \frac{3}{4} z^{-1}} \]\

Al sistema viene messo in ingresso il segnale:

\[ x[n] = \bigg(\frac{1}{3}\bigg)^n \cdot u[n] + u[-n -1] \]\

Si richiede di:

\begin{enumerate}
	\item[\textbf{1)}] Calcolare la risposta all’impulso $h[n]$ del sistema.
	\item[\textbf{2)}] Calcolare l’uscita $y[n]$.
	\item[\textbf{3)}] Dire se il sistema è stabile.
\end{enumerate}


\end{document}














