\documentclass[../main.tex]{subfiles}

\graphicspath{{\subfix{../images/}}}

\begin{document}

\newpage

%% BEGIN WRITING %%

\subsection{Esercitazione 4 - 4/12/2025}

% TODO INTRO ES-4 ENS

\subsubsection{Esercizio 1 - Risposta all’impulso del filtro numerico}

Calcolare la risposta all’impulso del filtro numerico specificato dalla seguente
equazione ricorsiva:

\[ y[n] = x[n] - x[n-1] + \frac{3}{2} \cdot y[n - 1]\]\

Iniziamo la risoluzione di questo esercizio andando a disegnare il circuito che implementa il filtro numerico rappresentato sopra.
Come possiamo vedere l'uscita del filtro è composta dalla sommatoria dello stesso segnale in ingresso da una sua copia ritardata di un campione e da una reazione amplificata del campione precedente.
Ottenute queste informazioni possiamo rappresentare il circuito equivalente del filtro numerico come di seguito in figura:

% TODO FOTO IIR

Fatto ciò possiamo andare a calcolare la risposta all’impulso del filtro numerico in due modi diversi: come prodotto di convoluzione oppure attraverso la trasformata Z. Di seguito andremo ad usare entrambi i modi.

Iniziamo ora andando ad usare la trasformata Z. 
Come prima cosa andiamo a fare la trasformata del filtro, ottenendo che:

\[ Z \big{ \{ }y[n]\big{ \{ } = Y(z) = X(z) - X(z) \cdot z^{-1} + \frac{3}{2} Y(z)\cdot z^{-1} \]\

Arrivati a questo punto possiamo andare a portare tutti i termini $Y(z)$ al primo membro effettuando un raccoglimento, si ottiene che:

\[ Y(z) \big[ 1 - \frac{3}{2}\cdot z^{-1} \big] = X(z) \big[ 1 - z^{-1} \big] \]\

Ricordando ora che:

\[ H(z) = \frac{Y(z)}{X(z)} \]\

Si ottiene che:

\[ H(z) = \frac{Y(z)}{X(z)}  = \frac{1 - z^{-1} }{1 - \frac{3}{2}\cdot z^{-1}} \]\

Separo il risultato ottenuto in due frazioni sottratte tra di loro nel modo seguente:

\[ H(z) = \frac{1 - z^{-1} }{1 - \frac{3}{2}\cdot z^{-1}} = \frac{1}{1 - \frac{3}{2}\cdot z^{-1}} - \frac{z^{-1}}{1 - \frac{3}{2}\cdot z^{-1}}\]\

Arrivati a questo punto possiamo andare ad antitrasformare il risultato ottenuto per avere la risposta all'impulso $h(n)$.
Come possiamo vedere, le frazioni ottenute sopra, sono simili tra loro a differenza di un termine $z^{-1}$ il quale introdurrà un ritardo di un campione.
A questo punto utilizzando la seguente relazione, disponibile sulle tavole andiamo ad antitrasformare:

\[ \frac{1}{1 - \alpha \cdot z^{-1}} = \alpha^n \cdot u[n]  \]\

Otteniamo che:

\[ \frac{1}{1 - \frac{3}{2} \cdot z^{-1}} = \bigg( \frac{3}{2}\bigg)^n  \cdot u[n]  \]\

Mentre che:

\[ \frac{z^{-1}}{1 - \frac{3}{2} \cdot z^{-1}} = \bigg( \frac{3}{2}\bigg)^{n-1}  \cdot u[n-1]  \]\

Otteniamo infine che:

\[ h(n) = \bigg(\frac{3}{2}\bigg)^n  \cdot u[n] - \bigg(\frac{3}{2}\bigg)^{n-1}  \cdot u[n-1] \]\

Possiamo inoltre riscrivere la soluzione ottenuta in funzione di $[n-1]$ andando an interpretare $u[n]$ nel seguente modo:

\[  u[n] = \delta[n] + u[n-1]  \]\

Si ottiene che:

\[ h(n) = \bigg(\frac{3}{2}\bigg)^n  \cdot \big[  \delta[n] + u[n-1]  \big] - \bigg(\frac{3}{2}\bigg)^{n-1}  \cdot u[n-1] \]\

Riscrivo la prima potenza in termini di $[n-1]$:

\[ h(n) = \bigg(\frac{3}{2}\bigg)\bigg(\frac{3}{2}\bigg)^{n-1} \cdot \big[  \delta[n] + u[n-1]  \big] - \bigg(\frac{3}{2}\bigg)^{n-1}  \cdot u[n-1]   \]\

Esplicito i prodotti:

\[ h(n) = \bigg(\frac{3}{2}\bigg)\bigg(\frac{3}{2}\bigg)^{n-1} \cdot  \delta[n] + \bigg(\frac{3}{2}\bigg)\bigg(\frac{3}{2}\bigg)^{n-1} \cdot  u[n-1]  - \bigg(\frac{3}{2}\bigg)^{n-1}  \cdot u[n-1]  \]\

A questo punto posso andare a semplificare il primo termine in quanto $\delta[n]$ è non nulla solo in $n = 0$. Quando la condizione di verifica le die frazioni si elidono lasciandoci la sola delta. Otteniamo che:

\[ h(n) = \delta[n] + \bigg(\frac{3}{2}\bigg)\bigg(\frac{3}{2}\bigg)^{n-1} \cdot  u[n-1]  - \bigg(\frac{3}{2}\bigg)^{n-1}  \cdot u[n-1]  \]\

Effettuo il raccoglimento al termine rimanente, ottengo che:

\[ h(n) = \delta[n] + \bigg( \frac{3}{2} - 1 \bigg) \bigg(\frac{3}{2}\bigg)^{n-1} \cdot  u[n-1] \]\

Semplificando e svolgendo i calcoli ottengo che:

\[ h[n] = \delta[n] +  \frac{1}{2} \cdot \bigg(\frac{3}{2}\bigg)^{n-1} \cdot  u[n-1] \]\

Ottenuto questo risultato andiamo ora ed effettuare il procedimento operando nel dominio del tempo discreto.

Come prima cosa ricordiamo che la risposta all'impulso altro non è che il segnale prodotto in uscita da un sistema, in questo caso il filtro, quando viene posta un'impulso al suo ingresso.
Inoltre poniamo come ipotesi che % FINIRE DUBBIO

Detto ciò e fatte le opportune premesse andiamo a sostituire $x[n]$ con la sequenza desiderata ovvero $\delta[n]$. Ricordiamo inoltre che che la reazione si trasforma anch'essa contribuendo a sua volta come risposta all'impulso. 

Otteniamo dunque che:

\[ h[n] = \delta[n] - \delta[n-1] + \frac{3}{2} \cdot h[n-1] \]\

Andiamo ora a calcolare la risposta all'impulso in modo ricorsivo come di seguito:

\[ h[0] = \delta[0] - \delta[-1] + \frac{3}{2}\cdot h[-1] = 1 - 0 + 0 = 1\]
\[ h[1] = \delta[1] - \delta[0] + \frac{3}{2}\cdot h[0] = 0 - 1 + \frac{3}{2} = \frac{1}{2}\]
\[ h[2] = \delta[2] - \delta[1] + \frac{3}{2}\cdot h[1] = 0 - 0 + \frac{3}{2} \cdot \frac{1}{2}\]
\[ h[3] = \delta[3] - \delta[2] + \frac{3}{2}\cdot h[2] = 0 - 0 + \bigg(\frac{3}{2}\bigg)^2 \cdot \frac{1}{2}\]


Dopo alcuni passaggi possiamo notare che con il proseguire dei passi il risultato risulta essere pari a:

\[ h[n] = \delta[n] + \frac{1}{2} \cdot \bigg(\frac{3}{2}\bigg)^{n-1} \]\

In particolare la sequenza $\delta[n]$ ha il solo scopo di portare il risultato ad uno con in $n = 0$.
Dobbiamo ora fare una sola ultima aggiunta in quanto nelle premesse avevamo ipotizzato che la risposta all'impulso fosse nulla per ogni campione $n < 0$. In altri termini abbiamo ipotizzato che il sistema fosse scarico. Per risolvere questo problema andiamo a moltiplicare quanto ottenuto per una sequenza gradino ritardata di un solo campione. 

Si ottiene che:

\[ h[n] = \delta[n] + \frac{1}{2} \cdot \bigg(\frac{3}{2}\bigg)^{n-1}  \cdot u[n-1]\]\

Arrivati a questo punto notiamo come il risultato ottenuto ora sia analogo a quello ottenuto in precedenza dimostrando la validità di entrambi i processi.

\subsubsection{Esercizio 2 - Analisi filtro FIR I}

Dato il filtro FIR:

\[ y[n] = x[n] - x[n-4]  \]\

\begin{enumerate}
	\item[\textbf{1)}] Calcolare e disegnare il modulo e la fase della funzione di trasferimento $H(f) = H(e^{2j\pi f})$
	\item[\textbf{2)}] Calcolare la sequenza in uscita dal filtro quando in ingresso abbiamo la sequenza: \[ x[n] = \cos\bigg[ \frac{\pi}{2} n \bigg] + \cos\bigg[ \frac{\pi}{4} n \bigg] \]\
	\item[\textbf{3)}] Giustificare il risultato di \textbf{2)} utilizzando il risultato di \textbf{1)}
\end{enumerate}


Iniziamo la risoluzione di questo esercizio andando a disegnare il circuito equivalente del filtro.
Riportiamo di seguito il risultato ottenuto.

% TODO FILTRO SIEGNO

A questo punto possiamo andare a calcolare la funzione di trasferimento utilizzando la relazione vista in precedenza:

\[ H(z) = \frac{Y(z)}{X(z)}\]\

Dove $Y(z)$ è la trasformata Z del filtro che è pari a:

\[ Z\big{\{} x[n] \big{\}} = Y(z) = X(z) = X(z) \cdot z^{-4} \]\

Otteniamo in modo immediato che:

\[  H(z) = 1 - z^{-4}  \]\

A questo punto non ci rimane che andare a calcolare la funzione di trasferimento alla frequenza richiesta:

\[ H(f) = H(e^{j2\pi f})  = 1-z^{-4}\big|_{e^{j2\pi f}} \]\

Si ottiene che:

\[ H(e^{j2\pi f}) = 1 - z^{-j8\pi f}  \]\

Possiamo riscrivere il risultato come somma di sinusoidi secondo le proprietà dei numeri complessi, otteniamo che:

\[ H(e^{j2\pi f}) = 1 - \cos(8\pi f) + j\cdot \sin(8\pi f) \]\

Così facendo abbiamo esplicitato parte reale e parte immaginaria che risultano essere rispettivamente i primi due termini e il terzo termine moltiplicato per l'unità immaginaria $j$.

A questo punto calcoliamo il modulo e la fase.
Calcolo il modulo quadro come somma dei quadrati delle parti reali ed immaginarie.

\[ \big| H(e^{j2\pi f}) \big|^2 = \big[1 - \cos(8\pi f)\big]^2 + \sin^2(8\pi f) \]

\[ \big| H(e^{j2\pi f}) \big|^2  = 1 - 2\cos(8\pi f) + \cos^2(8\pi f) +\sin^2(8\pi f)\]\

Semplificando le funzioni trigonometriche elevate a quadrato ottengo che:

\[ \big| H(e^{j2\pi f}) \big|^2 = 2 - 2\cos(8\pi f) \]\

Nel caso in cui si volesse ottenere il modulo basta effettuare la radice quadrata al risultato appena ottenuto come indicato di seguito:

\[ \big| H(e^{j2\pi f}) \big| = \sqrt{2 - 2\cos(8\pi f) }   \]\

Passando oltre andiamo ora a calcolare la fase come arcotangente del rapporto tra parte immaginaria e reale:

\[ \angle \; H(e^{j2\pi f}) =  \arctan\bigg( \frac{\sin(8\pi f)}{1 - \cos(8\pi f)}  \bigg)\]\

Andiamo ora a tracciare i grafici di modulo e fase appena ottenuti.

% TODO GRAFICO FASE E MODULO

Come possiamo vedere i grafico della fase si annulla in ogni punto in cui vale la relazione seguente:

\[ 2 - 2\cos(8\pi f) = 0 \]\

Ovvero dove la funzione $\cos(8\pi f)$ è pari ad uno. Questo succede in ogni $2\pi k$ punto. Sostituendo si ottiene che $f = \frac{k}{4}$ e che quindi il grafico avrà uno zero in ogni punto come $\pm \frac{1}{4}, \pm \frac{1}{2}, \pm \frac{3}{4},\; \dots $ e così via. Infine notiamo come il grafico del modulo sia traslato di 2 verso l'alto e moltiplicato del doppio in ampiezza.

Passando alla fase otteniamo il grafico visto in precedenza. In particolare notiamo come dove la fase è massima il modulo è nullo mentre dove la fase è nulla il modulo è massimo.\\

Andiamo infine a rispondere al secondo punto andando a calcolare l'uscita del filtro nel momento in cui si pone in ingresso la sequenza $x[n]$ vista prima.
Per calcolare la sequenza in ingresso andiamo a sostituire $x[n]$ con la sequenza fornita.
Applicando i ritardi opportuni si ottiene che:

\[ y[n] = \cos\bigg[ \frac{\pi}{2} n \bigg] + \cos\bigg[ \frac{\pi}{4} n \bigg] - \cos\bigg[ \frac{\pi}{2} (n-4)\bigg]  + \cos\bigg[ \frac{\pi}{2} (n-4) \bigg] \]\

Esplicito le moltiplicazioni all'interno delle funzioni trigonometriche:


\[ y[n] = \cos\bigg[ \frac{\pi}{2} n \bigg] + \cos\bigg[ \frac{\pi}{4} n \bigg] - \cos\bigg[ \frac{\pi}{2} n - 2\pi \bigg]  + \cos\bigg[ \frac{\pi}{2} n - 2\pi \bigg] \]\

Come possiamo vedere per gli ultimi due termini contengono la sottrazione di un valore pari a $2\pi$ il che non fa variare il loro valore. Possiamo dunque scrivere che:

\[ y[n] = \cos\bigg[ \frac{\pi}{2} n \bigg] + \cos\bigg[ \frac{\pi}{4} n \bigg] - \cos\bigg[ \frac{\pi}{2} n  \bigg]  + \cos\bigg[ \frac{\pi}{2} n  \bigg] \]\

Raccogliendo ed eliminando i termini opposti otteniamo che:

\[ y[n] =  2\cos\bigg[ \frac{\pi}{4} n \bigg] \]\


\subsubsection{Esercizio 3 - Analisi sistema LTI a tempo discreto I}


Si consideri un sistema LTI a tempo discreto, descritto dalla seguente relazione ingresso/uscita:

\[ y(n) = \alpha x(n - 1) + 2\beta y(n - 1) - \beta^2 y (n - 2) \]\

dove $\alpha$ e $\beta$ sono numeri reali.


Si richiede di:

\begin{enumerate}
	\item[\textbf{1)}] Disegnare lo schema circuitale del sistema.
	\item[\textbf{2)}] Calcolare la funzione di trasferimento $H(z)$ e discutere la stabilità del sistema al variare dei parametri $\alpha$ e $\beta$.
	\item[\textbf{3)}] Calcolare la risposta all’impulso $h(n)$ e la risposta in frequenza $H(e^{j2\pi f})$.
	\item[\textbf{4)}] Ponendo $\alpha = 1$ e $\beta = \sqrt(2)$, calcolare l’uscita l’uscita $y(n)$ quando all’ingresso è posto il segnale $x(n)$ ottenuto dal campionamento della sinusoide analogica $x(t) = cos(2\pi f_0t)$ con frequenza di campionamento pari al quadruplo della frequenza di Nyquist.
\end{enumerate}


Iniziamo la risoluzione di questo esercizio andando a disegnare il circuito equivalente implementato dal filtro preso in analisi.
Come possiamo vedere il filtro applica subito un ritardo al segnale in ingresso moltiplicandolo per la prima costante $\alpha$. In seguito sono presenti due reazioni rispettivamente ritardate di uno e due campioni e moltiplicate per le costanti $2\beta$ e $-\beta^2$ per poi essere successivamente sommate.
Otteniamo dunque il seguente circuito:

% TODO CIRCUITO 

\subsubsection{Esercizio 4 - Analisi sistema LTI a tempo discreto II}

Si consideri il sistema LTI discreto con la seguente relazione tra ingresso e uscita:

\[  y[n] - \frac{1}{4} y[n - 1] = x[n] + x[n - 1] - 2x[n - 2] \]\

Si richiede di:

\begin{enumerate}
	\item[\textbf{1)}] Studiare poli e zeri della funzione di trasferimento.
	\item[\textbf{2)}] Dire se il sistema è di tipo FIR o IIR.
	\item[\textbf{3)}] Dire se il filtro è a fase minima.
\end{enumerate}

Andiamo ora ad affrontare la risoluzione dell'esercizio proposto.
Come prima cosa per ottenere la funzione di trasferimento di cui dovremo studiarne zeri e poli andiamo a trasformare nel dominio della trasformata Z la relazione ingresso uscita del filtro. Otteniamo che:

\[ Y(z) - \frac{1}{4} Y(z)\cdot z^{-1}  = X(z) + X(z) \cdot z^{-1} -2X(z) \cdot z^{-2} \]\

Effettuiamo il raccoglimento:


\[ Y(z) \big[1 - \frac{1}{4} z^{-1}\big]  = X(z) \big[1 + z^{-1} -2z^{-1}\big]  \]\

Calcolo ora la funzione $H(z)$:

\[ H(z) = \frac{Y(z)}{X(z)} = \frac{1 + z^{-1} -2z^{-1} }{1 - \frac{1}{4} z^{-1}} \]\

Raccolgo il termine $z^{-2}$ al numeratore e $z^{-1}$ al numeratore per riscrivere la relazione in termini di $z$. Ottengo dunque che:

\[ H(z) = \frac{z^{-2}}{z^{-1}} \cdot \frac{z^2 + z - 2}{z - \frac{1}{4}} = \frac{z^2 + z - 2}{z \cdot (z - \frac{1}{4})} \]\

A questo punto possiamo calcolare gli zeri ed i poli. In particolare gli zeri saranno dati da:

\[z^2 + z - 2 = 0  \hspace{10pt} \rightarrow \hspace{10pt} \begin{cases}
	z = 1\\
	z = 2
\end{cases}\]\

I poli invece saranno dati da:

\[ z \cdot \bigg( z - \frac{1}{4} \bigg)  = 0 \hspace{10pt}  \rightarrow \hspace{10pt} \begin{cases}
	z = 0\\
	z = \frac{1}{4}
\end{cases}\]\

A questo punto possiamo andare a riportare i valori ottenuti sulla circonferenza di raggio unitario.
In particolare prendiamo in considerazione il polo con il valore maggiore per definire la regione di convergenza ROC.
Si avrà dunque che $|z| > \tfrac{1}{4}$.
Riportiamo anche di seguito il grafico generato in MATLAB.

\begin{figure}[h!]
\centering
\includegraphics[width=.75\textwidth]{images/e4.4.png}
\end{figure}

In aggiunta a quanto detto prima possiamo andare a dire che la relazione ingresso uscita è causale in quanto fa riferimento sempre a valori passati della sequenza.

Ora procediamo con il secondo punto dove dobbiamo dire se il sistema è di tipo FIR o IIR.
Come possiamo vedere dalla relazione ingresso uscita è presente una retroazione che, seppur moltiplicata per un fattore minore di uno, rende la risposta del filtro infinita.
Arriviamo dunque alla conclusione che il sistema sia di tipi IIR.

Passiamo all'ultimo punto dove andiamo a verificare se il sistema è a fase minima.
Per effettuare questa verifica riprendiamo la relazione $H(z)$ andandola ad invertire per trovarne l'opposto ovvero $H^{-1}(z)$.
Fatto ciò possiamo verificare se quel dato sistema sia stabile nel caso in cui lo sia possiamo dire che il sistema di partenza è a fase minima.
In particolare abbiamo che:

\[  H(z) = \frac{z^2 + z - 2}{z \cdot (z - \frac{1}{4})}   \]\


\[  H^{-1}(z) = \frac{z \cdot (z - \frac{1}{4})}{z^2 + z - 2}   \]\

Effettuando questa operazione notiamo come valori di poli e zeri siano invertiti.
In particolare notiamo come i valori ricadano al di fuori della circonferenza unitaria e che quindi essendo $|z| > 1$ il sistema non risulta essere stabile.
Come detto in precedenza se il sistema non è stabile non si ha fase minima.
Riportiamo per completezza il grafico dei poli e zeri di $H^{-1}(z)$:

\begin{figure}[h!]
\centering
\includegraphics[width=.75\textwidth]{images/e4.4-1.png}
\end{figure}

\subsubsection{Esercizio 5 - Analisi filtro FIR II}

Si consideri il filtro di tipo FIR con $H(z)$ come riportata di seguito.

\[H(z) = (1 - 0.9e^{j0.6\pi} z^{-1})(1 - 0.9 e^{-j0.6\pi} z^{-1} )(1 - 1.25 e^{j0.8\pi} z^{-1})(1 - 1.25 e^{-j0.8\pi} z^{-1}) \]\

Si richiede di:

\begin{enumerate}
	\item[\textbf{1)}] Studiare poli e zeri di $H(z)$.
	\item[\textbf{2)}] Studiare la stabilità del sistema inverso ${H^{-1}(z)}$.
\end{enumerate}

Cominciamo ora con la risoluzione dell'esercizio proposto. Come prima cosa possiamo notare come ci sia possibile effettuare il raccoglimento del termine $z^{-1}$. Si ottiene che:

\[ H(z) = \frac{1}{z^{4}} \cdot  (1 - 0.9e^{j0.6\pi})(1 - 0.9 e^{-j0.6\pi})(1 - 1.25 e^{j0.8\pi})(1 - 1.25 e^{-j0.8\pi}) \]\  

Portando il termine raccolto al denominatore possiamo subito individuare i poli di $H(z)$, essi sono quattro e tutti pari a zero.

Passando oltre notiamo come i valori dei moduli degli esponenziali complessi contenuti all'interno delle parentesi siano uguali. Possiamo andare a calcolare gli zeri di $H(z)$ a coppie. Otteniamo che gli zeri saranno pari a $0.9e^{\pm j 0.6 \pi}$ ed a $1.25e^{\pm j 0.8 \pi}$. 
Notiamo infine come le due coppie siano composte da numeri complessi coniugati tra loro.
Infine riportiamo il grafico come fatto anche in precedenza:


\begin{figure}[h!]
\centering
\includegraphics[width=.75\textwidth]{images/e4.4-2.png}
\end{figure}

\subsubsection{Esercizio 6 - Analisi funzione di trasferimento}

Data la seguente funzione di trasferimento:

\[ H(z) = \frac{z^{-1} - \frac{1}{3}}{1 - \frac{1}{3} z^{-1}} \]\


Si richiede di:

\begin{enumerate}
	\item[\textbf{1)}] Studiare zeri e poli e discutere la stabilità.
	\item[\textbf{2)}] Calcolare il modulo della funzione di trasferimento $\big|H(e^{j2\pi f})\big|$
\end{enumerate}

\subsubsection{Esercizio 7 - Funzione di trasferimento dato un segnale}

La funzione di trasferimento di un sistema numerico vale:

\[ H(z) = \frac{1 - z^{-1} }{1 + \frac{3}{4} z^{-1}} \]\

Al sistema viene messo in ingresso il segnale:

\[ x[n] = \bigg(\frac{1}{3}\bigg)^n \cdot u[n] + u[-n -1] \]\

Si richiede di:

\begin{enumerate}
	\item[\textbf{1)}] Calcolare la risposta all’impulso $h[n]$ del sistema.
	\item[\textbf{2)}] Calcolare l’uscita $y[n]$.
	\item[\textbf{3)}] Dire se il sistema è stabile.
\end{enumerate}

\end{document}














