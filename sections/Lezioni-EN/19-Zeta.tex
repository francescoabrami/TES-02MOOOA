\documentclass[../main.tex]{subfiles}

\graphicspath{{\subfix{../images/}}}

\begin{document}

\newpage

\section{Trasformata Z}

\subsection{Introduzione}

Iniziamo ora questa nuova sezione andando ad introdurre una nuova tipologia di trasformata ovvero quella zeta. In particolare possiamo dire fin da subito che la trasformata zeta non è altro che uno strumento matematico che serve per studiare sequenze discrete, cioè definite su interi: $x[0], x[1], \dots , x[n]$,  nello stesso modo in cui la trasformata di Laplace studia i segnali continui. 
In breve la  trasformata zeta prende una sequenza e la trasforma in una funzione di una variabile complessa $z$.

\begin{definition*}
Si definisce trasformata zeta di una sequenza la seguente formula dove z è una generica variabile complessa definita come di seguito.
\end{definition*}

\[ X(z) = Z\big[x[n]\big] = \sum_{n = -\infty}^{+\infty} x[n] \cdot z^{-n}  \]\

In particolare definiamo $z$ come:

\[ z = \rho \cdot e^{j\omega} \]\

Possiamo duque riscrivere che:

\[ X(z) = Z\big[x[n]\big] = \sum_{n = -\infty}^{+\infty} x[n] \cdot (\rho \cdot e^{j\omega})^{-n} = \sum_{n = -\infty}^{+\infty} x[n] \cdot \rho^{-n} \cdot e^{-j\omega n}\]\

Data ora la definizione possiamo dire che $X(z)$ non è altro che una differente rappresentazione del segnale numerico preso in analisi.
In particolare possiamo dire che: il coefficiente del generico termine $z^{-n}$ è il valore del segnale all’istante $n$ e che l’esponente di $z$ contiene l’informazione temporale di cui abbiamo bisogno per identificare i campioni temporali del segnale in quanto è in esso che è presente la variabile tempo discreto utilizzata per i segnali non continui nel tempo.

Come possiamo vedere dopo questa prima occhiata siamo ora in grado di convertire una sequenza temporale in una funzione complessa così che sia più facile da trattare, risolvere ed analizzare. In un certo senso, come detto prima, la Z-transform altro non fa, nel tempo discreto, quello che faceva Laplace nel tempo continuo.
Vista ora questa similitudine possiamo andare ad analizzare le similitudini tra la trasformata Z e quella di Laplace.

\subsubsection{Analogia con la trasformata di Laplace}

Andiamo ora a vedere le analogie delle due trasformate ma come prima cosa, anche se potrebbe risultare un po' poco intuitivo, andiamo a riportare le formula della trasformata di Laplace e Fourier.

\begin{definition*}
Si definisce trasformata di Laplace
\end{definition*} 

\[ X(s) = \int_{-\infty}^{+\infty} x(t) \cdot e^{-st}\; dt \hspace{20pt} \unit{con} \hspace{5 pt} s = \sigma + j2\pi f_a \]\

\begin{definition*}
Si definisce trasformata di Fourier
\end{definition*} 

\[ X(f_a) = \int_{-\infty}^{+\infty} x(t) \cdot e^{-j2\pi f_a t} \; dt\]\

Come possiamo notare le due trasformate, considerate nella loro forma estesa, sono molto simili e differiscono solo per un termine $\sigma$ che in quella di Fourier assume il valore zero.
In altri termine nella trasformata di Laplace si utilizza $s = \sigma + j\Omega $, dove $\Omega = 2\pi f_a$, il quale corrisponde ad un generico numero complesso mentre in quella di Fourier viene utilizzata una variabile puramente immaginaria pari a $j\Omega$.\\

Vista questa similitudine possiamo giungere alla conclusione che la trasformata di Fourier altro non sia che una una particolarizzazione della trasformata di Laplace: quest’ultima è definita nel piano complesso (non su tutto, solo nella regione di convergenza), mentre la trasformata di Fourier solo sull’asse immaginario (sottoinsieme del piano complesso). Come possiamo notare quanto appena dedotto coincide con il ragionamento fatto sopra riguardo il complesso generico $s$ corrispondente all'intero piano di Argand-Gauss e il numero immaginario puro $j\Omega$ corrispondente al solo asse immaginario.\\

Possiamo in definitiva andare a tracciare una rappresentazione di quanto appena detto nell'immagine sottostante. In figura vengono semplicemente mostrati le regioni in cui vivono, o meglio, sono definite le due trasformate.

% TODO FIGURA


\subsection{DFT e trasformata Z}


\subsection{Regione di convergenza - ROC}

\subsection{Tipologie di sequenza}

\subsubsection{Termine causale e anticausale}
\subsubsection{ROC di sequenze unilatere}
\subsubsection{ROC di sequenze bilatere}
\subsubsection{ROC di sequenze a supporto finito}

\subsection{Trasformate Z razionali}
\subsubsection{Poli e zeri della trasformata Z}

\subsection{Sequenze bilatere e non a supporto illimitato}












	



\end{document}
















