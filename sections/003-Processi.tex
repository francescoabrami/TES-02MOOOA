\documentclass[../main.tex]{subfiles}

\graphicspath{{\subfix{../images/}}}

\begin{document}

\newpage

%% BEGIN SUBFILE %%

\section{Processi Casuali}


\subsection{Ripasso di teoria della probabilità e variabili casuali}

\subsubsection{Spazio campione}

\subsubsection{Probabilità congiunta}
\subsubsection{Probabilità condizionata}

\subsubsection{Teorema di Bayes}

\subsubsection{Variabile casuale}
\subsubsection{Funzione di distribuzione cumulativa}
\subsubsection{Densità di probabilità}
\subsubsection{Insiemi di variabili casuali}

\subsection{Distribuzione cumulativa condizionate}

\subsubsection{Valore atteso e momenti}

\subsubsection{Momenti centrali}

\subsubsection{Combinazione lineare di variabili casuali}

\subsubsection{Funzione caratteristica}

\subsubsection{Variabile Gaussiana}
\subsubsection{Densità di probabilità e istogrammi}
\subsubsection{Teorema limite centrale}

\subsection{Canale di comunicazione discreto}
\subsubsection{Caratterizzazione probabilistica del canale di comunicazione discreto}
\subsubsection{Probabilità di errore}
\subsubsection{Attendibilità del canale discreto e canale BSC con sorgente simmetrica}

\subsection{Introduzione ai processi casuali}

\subsubsection{Esempi di applicazione}
\subsubsection{Processi casuali e sistemi}		

\subsection{Definizione e tipologia di processi casuali}

\subsubsection{Descrizione probabilistica}

\subsubsection{Gaussian random walk}

\subsection{Descrizione di un processo casuale}

\subsubsection{Media}

\subsubsection{Autocorrelazione}

\subsubsection{Caratterizzazione tramite media e autocorrelazione}

\subsection{Processi stazionari}

\subsubsection{Introduzione ai processi stazionari e definizione di stazionarietà}

\subsubsection{Intepretazione stazionarietà}

\subsubsection{Processi stazionari in senso stretto e WSS}

\subsubsection{Proprietà e comportamento dell'autocorrelazione per processi WSS}























%% END SUBFILE %%

\end{document}