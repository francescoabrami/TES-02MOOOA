\documentclass[11pt]{article}

%% THIS SECTION IS FOR PACKAGES %%

\usepackage{makeidx}
\usepackage[italian]{babel}
\usepackage{floatflt}
\usepackage{titlesec}
\usepackage{listings}
\usepackage{amsmath}
\usepackage{amsthm}
\usepackage{amsfonts}
\usepackage{hyperref}
\usepackage{graphics}
\usepackage{wrapfig}
\usepackage[utf8]{inputenc}
\usepackage{subfiles}
\usepackage[version=3]{mhchem}
\usepackage{tikz}
\usepackage{pgfplots}
\usepackage{amssymb}
\usepackage{textcomp} 
\usepackage{enumitem}
\usepackage{blindtext}

%% SECTION FOR CUSTOM COMMAND %%

\newcommand{\unit}[1]{\ensuremath{\, \mathrm{#1}}}

%% SECTION FOR CUSTOM PATHS %%

\graphicspath{ {./images/} }

%% LAYOUT SECTION %%

\textwidth 15.5 cm
\textheight 23.0 cm
\topmargin -0.75 cm
\oddsidemargin 0.5 cm
\linespread{1.1}

%% TITLE INFORMATION AND VERSIONING %%

\title{\textbf{Teoria ed Elaborazione dei Segnali \\  02MOOOA \\ Prof.ssa Gabriella Bosco \\ Prof. Fabio Dovis}}
\author{Francesco Abrami - s310570}
\date{A.A. 2025 - 2026 \\ \vspace{10pt} \small {Versione: 1.11.0 - 07/11/2025}}

%% DISPLAY TITLE %%

\begin{document}

\maketitle
\thispagestyle{empty}

%% DISPLAY PICTURE OR LOGO ON FIRST PAGE %%

\begin{figure}[h!]
\centering
\includegraphics[width=.75\textwidth]{images/PoliTO.jpg}
\end{figure}

%% DISPLAY TABLE OF CONTENTS ON A NEW PAGE %%

\newpage
\tableofcontents

%% WRITING AND SUBFILE CALL SECTION %%


\subfile{sections/00-Generale.tex} % DONE

\subfile{sections/LEZIONI-TS/01-ALG.tex}
\subfile{sections/LEZIONI-TS/02-Segnali.tex}
\subfile{sections/LEZIONI-TS/03-Analogici.tex}
\subfile{sections/LEZIONI-TS/04-Vettori.tex}
\subfile{sections/LEZIONI-TS/05-Fourier.tex}
\subfile{sections/LEZIONI-TS/06-Sistemi.tex}
\subfile{sections/LEZIONI-TS/07-Blocchi.tex}
\subfile{sections/LEZIONI-TS/08-Periodici.tex}
\subfile{sections/LEZIONI-TS/09-Correlazione.tex}
\subfile{sections/LEZIONI-TS/10-Spettro.tex}
\subfile{sections/LEZIONI-TS/11-DAC.tex}

\subfile{sections/LEZIONI-EN/12-Numerici.tex}
\subfile{sections/LEZIONI-EN/13-Sequenze.tex}

\subfile{sections/Esercitazioni-TS/Esercitazione-1.tex}
\subfile{sections/Esercitazioni-TS/Esercitazione-2.tex}
\subfile{sections/Esercitazioni-TS/Esercitazione-3.tex}
\subfile{sections/Esercitazioni-TS/Esercitazione-4.tex}
\subfile{sections/Esercitazioni-TS/Esercitazione-5.tex}
\subfile{sections/Esercitazioni-TS/Esercitazione-6.tex}
\subfile{sections/Esercitazioni-TS/Esercitazione-7.tex}



\end{document}
